\chapter{Practical Applications of CNNs}

Convolutional Neural Networks (CNNs) have become the standard for a wide range of tasks in artificial intelligence, especially in the field of computer vision. 
Their ability to automatically extract hierarchical features from raw data has enabled breakthroughs across domains such as image recognition, object detection, medical imaging, and autonomous systems \cite{li2021survey, zhao2024review}.
\\
In this section, we explore the most relevant applications of CNNs, highlighting how the architectures presented in Section~\ref{tab:cnn_comparison} have been successfully deployed in real-world scenarios.

\section{Image Classification}

One of the earliest and most impactful applications of CNNs has been \textbf{image classification}, where the goal is to assign a label to an entire image. 
Starting from LeNet-5's success on MNIST to AlexNet's breakthrough on the ImageNet dataset, CNNs have consistently pushed the boundaries of classification performance.
\\
Modern architectures such as VGG, ResNet, and EfficientNet achieve top-1 accuracies above 90\% on ImageNet, making CNNs the backbone of most commercial image recognition systems \cite{li2021survey}. 
These capabilities power numerous services, from automatic photo tagging on social media platforms to visual search engines and large-scale multimedia retrieval systems.

\section{Object Detection and Localization}

While classification focuses on identifying what is present in an image, \textbf{object detection} extends the task to also determining \textit{where} objects are located.
CNN-based models such as R-CNN, Faster R-CNN, YOLO, and SSD leverage feature maps to predict bounding boxes and class labels simultaneously. 
\\
Thanks to these advances, CNNs are now widely used in applications such as:

\begin{itemize}
    \item \textbf{Autonomous driving}: detecting pedestrians, vehicles, and traffic signs in real time.
    \item \textbf{Security systems}: monitoring surveillance footage for abnormal activities.
    \item \textbf{Retail analytics}: recognizing products and tracking customer behavior.
\end{itemize}

According to Zhao et al. \cite{zhao2024review}, the integration of CNNs with attention mechanisms has further improved detection accuracy, especially in challenging environments.

\section{Semantic and Instance Segmentation}

Another critical application of CNNs is \textbf{image segmentation}, which involves partitioning an image into regions corresponding to different objects or classes.
Two main approaches are commonly distinguished:

\begin{itemize}
    \item \textbf{Semantic segmentation}: assigns a class label to every pixel, grouping identical objects under the same label.
    \item \textbf{Instance segmentation}: goes a step further by differentiating between individual objects of the same class.
\end{itemize}

Architectures such as U-Net, Mask R-CNN, and DeepLab have become standard tools in this field \cite{zhao2024review}. 
Applications range from autonomous navigation to industrial quality control and satellite image analysis.

\section{Medical Imaging}

Medical image analysis has benefited enormously from the adoption of CNNs, which are now routinely used in:

\begin{itemize}
    \item Detecting tumors and lesions in MRI and CT scans.
    \item Classifying skin diseases based on dermoscopic images.
    \item Automating radiological workflows by assisting with diagnosis.
\end{itemize}

According to Li et al. \cite{li2021survey}, CNN-based systems in medical imaging have achieved human-level accuracy in several diagnostic tasks. 
These advances accelerate clinical decision-making and reduce the workload of healthcare professionals.

\section{Autonomous Driving and Robotics}

Self-driving cars and autonomous robots rely heavily on CNNs to interpret their surroundings. 
Through a combination of image classification, object detection, and segmentation, CNNs enable real-time understanding of complex environments \cite{zhao2024review}. 
\\
For instance, Tesla’s Autopilot and Waymo’s self-driving systems integrate CNNs to detect pedestrians, vehicles, and obstacles, as well as to understand lane markings and traffic lights.

\section{Beyond Computer Vision}

Although CNNs were originally developed for visual tasks, they have been successfully applied beyond computer vision:

\begin{itemize}
    \item \textbf{Natural Language Processing}: character-level CNNs for text classification and sentiment analysis.
    \item \textbf{Audio Processing}: spectrogram-based CNNs for speech recognition and music genre classification.
    \item \textbf{Multimodal Models}: combining CNNs with Transformers for tasks that integrate vision and language, such as image captioning.
\end{itemize}