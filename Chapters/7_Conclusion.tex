\chapter{Conclusions}

Convolutional Neural Networks have fundamentally transformed the field of artificial intelligence, driving major breakthroughs in computer vision and beyond.  
From the pioneering work of LeNet-5 \cite{lecun1998gradient} to the revolutionary impact of AlexNet \cite{krizhevsky2012imagenet} and the subsequent advances brought by VGG, ResNet, and EfficientNet \cite{simonyan2014very, he2015deep, tan2019efficientnet}, CNNs have evolved into powerful models capable of extracting rich hierarchical representations from complex data.
\\
As discussed throughout this work, CNNs underpin a vast range of real-world applications, from image classification and object detection to medical diagnostics and autonomous systems \cite{li2021survey, zhao2024review}.  
Their capacity to automatically learn meaningful features from raw data has made them a central technology in modern AI.
\\
However, despite these successes, several open challenges remain.  
Future research must address the high computational demands of deep models, improve their interpretability, and ensure robustness against adversarial inputs.  
At the same time, the emergence of alternative paradigms such as Vision Transformers points towards a future where CNNs coexist with, or are complemented by, new architectural approaches \cite{zhao2024review}.
\\
In conclusion, CNNs represent both a milestone and a starting point in the development of intelligent systems.  
Their evolution reflects the broader trajectory of deep learning: from simple architectures designed for digit recognition to highly optimized models powering state-of-the-art applications across diverse domains.  
By combining CNNs with emerging paradigms and addressing their current limitations, the next generation of models promises to deliver even greater capabilities, shaping the future of artificial intelligence.