\chapter{Challenges and Future Directions}

Despite their remarkable success, Convolutional Neural Networks (CNNs) still face several open challenges that limit their broader applicability. 
\\
As highlighted in recent surveys \cite{li2021survey, zhao2024review}, current research focuses on improving the efficiency, robustness, and interpretability of these models while exploring new paradigms that may complement or even surpass CNNs.

\section{Computational and Memory Constraints}

Modern CNN architectures, such as VGG and EfficientNet, often involve tens of millions of parameters and require significant computational resources during both training and inference. 
This makes their deployment difficult on mobile devices, embedded systems, and other environments where hardware is limited.  
\\
To address this limitation, researchers are developing approaches such as model compression, parameter pruning, and quantization to reduce memory usage and computational cost without sacrificing accuracy. Another promising direction is knowledge distillation, where smaller “student” models are trained to mimic the behavior of larger, more complex “teacher” networks, making CNNs more suitable for real-time applications \cite{zhao2024review}.

\section{Interpretability and Explainability}

Although CNNs achieve state-of-the-art results on many vision tasks, they are often criticized for being black-box models.
\\
Understanding why a model makes a particular prediction is crucial in domains where transparency is required, such as healthcare, finance, and autonomous driving.  
Recent work has introduced techniques like saliency maps, class activation maps (CAMs), and feature attribution methods to provide visual and quantitative explanations of network decisions. 
\\
As Li et al. \cite{li2021survey} observe, improving interpretability will be essential to fostering trust in AI systems, especially in sensitive and safety-critical contexts.

\section{Robustness and Adversarial Attacks}

Another major challenge is robustness. CNNs perform well on clean datasets but are vulnerable to adversarial attacks — small, carefully designed perturbations that can mislead the model into making incorrect predictions with high confidence.  
This vulnerability poses significant risks in applications like autonomous driving, biometric authentication, and medical diagnostics.  
\\
To mitigate this issue, researchers are exploring adversarial training, defensive distillation, and certified defenses, but achieving robust and secure CNNs remains an open problem \cite{zhao2024review}.

\section{Beyond CNNs: The Rise of Vision Transformers}

While CNNs dominate computer vision, a new class of models known as Vision Transformers (ViTs) has recently gained attention.  
Unlike CNNs, which rely on local receptive fields and convolutional operations, ViTs use self-attention mechanisms to model global dependencies across an entire image.  
\\
This shift has led to significant improvements in benchmark performance, especially when large datasets are available.  
However, ViTs often require more computational resources, which has spurred interest in hybrid architectures combining the strengths of CNNs and Transformers \cite{zhao2024review}.

\section{Future Perspectives}

Looking ahead, research is likely to focus on making CNNs more lightweight, interpretable, and robust.  
Efforts to optimize architectures for mobile and edge devices will make deep learning more accessible in real-time applications.  
\\
At the same time, developing transparent and explainable models will be key to gaining acceptance in sensitive areas such as healthcare and security.  
Finally, the integration of CNNs with emerging paradigms — including Vision Transformers and graph neural networks — suggests a future where hybrid architectures become the standard \cite{li2021survey, zhao2024review}.