% Define blue color typical of polimi
\definecolor{bluepoli}{cmyk}{0.4,0.1,0,0.4}

% Custom theorem environments
\declaretheoremstyle[
  headfont=\color{bluepoli}\normalfont\bfseries,
  bodyfont=\color{black}\normalfont\itshape,
]{colored}

% Set-up caption colors
\captionsetup[figure]{labelfont={color=bluepoli}} % Set colour of the captions
\captionsetup[table]{labelfont={color=bluepoli}} % Set colour of the captions
\captionsetup[algorithm]{labelfont={color=bluepoli}} % Set colour of the captions

\theoremstyle{colored}
\newtheorem{theorem}{Theorem}[chapter]
\newtheorem{proposition}{Proposition}[chapter]

% Enhances the features of the standard "table" and "tabular" environments.
\newcommand\T{\rule{0pt}{2.6ex}}
\newcommand\B{\rule[-1.2ex]{0pt}{0pt}}

% Pseudo-code algorithm descriptions.
\newcounter{algsubstate}
\renewcommand{\thealgsubstate}{\alph{algsubstate}}
\newenvironment{algsubstates}
  {\setcounter{algsubstate}{0}%
   \renewcommand{\STATE}{%
     \stepcounter{algsubstate}%
     \Statex {\small\thealgsubstate:}\space}}
  {}

% New font size
\newcommand\numfontsize{\@setfontsize\Huge{200}{60}}

% Title format: chapter
\titleformat{\chapter}[hang]{
\fontsize{30}{20}\selectfont\bfseries\filright}{\textcolor{bluepoli} \thechapter\hsp\hspace{2mm}\textcolor{bluepoli}{|   }\hsp}{0pt}{\huge\bfseries \textcolor{bluepoli}
}

% Title format: section
\titleformat{\section}
{\color{bluepoli}\normalfont\Large\bfseries}
{\color{bluepoli}\thesection.}{1em}{}

% Title format: subsection
\titleformat{\subsection}
{\color{bluepoli}\normalfont\large\bfseries}
{\color{bluepoli}\thesubsection.}{1em}{}

% Title format: subsubsection
\titleformat{\subsubsection}
{\color{bluepoli}\normalfont\large\bfseries}
{\color{bluepoli}\thesubsubsection.}{1em}{}

% Shortening for setting no horizontal-spacing
\newcommand{\hsp}{\hspace{0pt}}

\makeatletter
% Renewcommand: cleardoublepage including the background pic
%\renewcommand*\cleardoublepage{%
%  \clearpage\if@twoside\ifodd\c@page\else
%  \null
%  \AddToShipoutPicture*{\BackgroundPic}
%  \thispagestyle{empty}%
%  \newpage
%  \if@twocolumn\hbox{}\newpage\fi\fi\fi}
\makeatother

%For correctly numbering algorithms
\numberwithin{algorithm}{chapter}

\usepackage{etoolbox}
\makeatletter
% Se nel wrapper della classe c'è un \vspace{-1cm}, sostituiscilo:
\patchcmd{\polimichapter}{\vspace{-1cm}}{\vspace{0pt}}{}{}
\patchcmd{\polimichapterstar}{\vspace{-1cm}}{\vspace{0pt}}{}{}
\makeatother

\usepackage{titlesec}
% \titlespacing*{\chapter}{<rientro>}{<spazio PRIMA>}{<spazio DOPO>}
\titlespacing*{\chapter}{0pt}{10pt}{8pt} % es.: niente spazio sopra, 20pt sotto

% --- Niente salto pagina all'inizio capitolo, ma mantiene numerazione/TOC ---
\usepackage{etoolbox}
\makeatletter
% rimuove il clearpage/cleardoublepage dentro \chapter (stile 'book')
\patchcmd{\chapter}{\if@openright\cleardoublepage\else\clearpage\fi}{}{}{}
% se la classe ha salvato il \chapter in \latexchapter e lo richiama via wrapper, patchiamo anche quello (se esiste)
\patchcmd{\latexchapter}{\if@openright\cleardoublepage\else\clearpage\fi}{}{}{}
\makeatother
